\documentclass[11pt,a4 paper,one side]{article}
\usepackage{amsmath,amssymb,graphicx}
\usepackage{ctex}  
\usepackage[colorlinks=true,linkcolor=red,citecolor=red,filecolor=magenta,urlcolor=cyan]{hyperref}
\usepackage{bookmark}
\usepackage{fontspec}
\setmainfont{Times New Roman}
\usepackage{xcolor}
\usepackage{geometry}
\geometry{a4paper, left=2.5cm, right=2.5cm, top=2.5cm, bottom=2.5cm}
\title{Allen-Cahn方程数值格式设计}
\date{\today}
\author{蒋鹏}
\begin{document}
\maketitle
\tableofcontents
\section{问题描述}
我们欲求解下述方程\begin{equation}
    \frac{\partial u}{\partial t}- \varepsilon^2 \Delta u + f(u) = 0;
\end{equation}
\\ 求解区域为$\Omega = [-1,1]^2$。初值条件为$u(x,y,0) = u_0(x,y)$。其中$f(u)$是非线性$f(u) = \log{(\frac{u}{1-u})}+\theta (1-2u) $.
\par 定义$g(u) = \log{(\frac{u}{1-u})}$,$G(u) = u\log{u}+(1-u)\log{(1-u)}$,$F(u) = u\log{u}+(1-u)\log{(1-u)}+\theta (u-u^2)$,
则$F'(u)=f(u)$,$G'(u)=g(u)$。
\section{数值算法}
\subsection{凸分裂格式}

利用时间向后Euler,空间五点差分格式 \begin{equation}
    \frac{U^{n+1}-U^n}{dt} - \varepsilon^2 \Delta_h U^{n+1} + f(U^{n+1})=0;
\end{equation}
做一步凸分裂,得到格式\begin{equation}
    \frac{U^{n+1}-U^n}{dt} - \varepsilon^2 \Delta_h U^{n+1} + g(U^{n+1}) +\theta (1-2U^n)=0;
\end{equation}
\subsection{基于泛函的ADMM格式}
定义泛函\begin{equation}
    J(U) = \int_{\Omega} ( \frac{\|U-U^n\|^2}{2dt}+  \frac{1}{2}\varepsilon^2 |\nabla U|^2  + G(U)  + \theta <U,1-2U^n>) dx;
\end{equation} 
于是格式等价于\begin{equation}
    U^{n+1} = \arg \min_{U \in V_h} J(U);
\end{equation}
定义\begin{equation}
    F_1(U) = \int_{\Omega} (\frac{\|U-U^n\|^2}{2dt} +  \frac{1}{2}\varepsilon^2 |\nabla U|^2 )dx
\end{equation}
和\begin{equation}
    F_2(U) =  \int_{\Omega} (G(U)  + \theta <U,1-2U^n> ) dx
\end{equation}
此时,原格式等价于\begin{equation}
    \begin{cases}
    \min_{U_1,U_2} F_1(U_1) + F_2(U_2)\\
     s.t. \quad U_1=U_2;
    \end{cases}
\end{equation}
令$Y$为Lagrange乘子,设置二次罚函数\begin{equation}
    L(U_1,U_2;Y) = F_1(U_1) + F_2(U_2) + <Y,U_1-U_2> + \frac{\rho}{2}\|U_1-U_2\|^2;
\end{equation}
得到ADMM格式\begin{equation}
    \begin{cases}
        U_1^{k+1} = \arg \min_{U_1} L(U_1,U_2^k;Y^k) \\
        U_2^{k+1} = \arg \min_{U_2} L(U_1^{k+1},U_2;Y^k) \\
        Y^{k+1} = Y^k + \tau \rho (U_1^{k+1}-U_2^{k+1})
    \end{cases}
\end{equation}
第一个子问题有显式解:\begin{equation}
    (\frac{1}{dt}-\varepsilon^2 \Delta_h + \rho)U_1^{k+1} = \frac{1}{dt}u^n + \rho U_2^k - Y^k;
\end{equation}
于是可以用CG或FFT进行求解,数值实验发现FFT更高效。
\par 第二个子问题的解满足\begin{equation}
    \log (\frac{U_2^{k+1}}{1-U_2^{k+1}})+\theta (1-2U^n) - Y^k -\rho (U_1^{k+1}-U_2^{k+1})=0;
\end{equation}
注意此方程可以视为$Nx*Ny$个一维问题,在每个网格点上,
\begin{equation}
    \log (\frac{u_2^{k+1}}{1-u_2^{k+1}})+\theta (1-2u^n) - y^k -\rho (u_1^{k+1}-u_2^{k+1})=0;
\end{equation}
可以用一维Newton法进行求根。
\\罚函数因子$\rho$进行自适应动态调节。
\section{高维Newton迭代}
仍然基于凸分裂格式,视其为高维非线性方程求根问题:在这里令\begin{equation}
b=\frac{1}{dt}U^n-\theta(1-2U^n)
\end{equation}
和矩阵\begin{equation}
A = \frac{1}{dt}-\varepsilon^2 \Delta_h
\end{equation}
于是凸分裂格式等价于求非线性方程$AU^{n+1}+g(U^{n+1})-b=0$的根。
\par Newton迭代格式\begin{equation}
    (A+g'(U_{k}))(U_{k+1}-U_k) = -AU_k-g(U_k)+b
\end{equation}
即\begin{equation}
    (A+g'(U_k))U_{k+1}=g'(U_k)U_k-g(U_k)+b
\end{equation}
注意$A$是对称正定矩阵,$g'(U_k)$是对角元均为正数的对角矩阵,进而该方程可以用CG进行求解。
\section{数值算例}
\subsection{算例一:随机初值}取初值函数\begin{equation}
u_0(x,y)=0.01 + 0.98 * (\text{(double)rand()} / (\text{RAND\_MAX} + 1.0))
\end{equation}
和参数$\theta = 4.0$.
\subsection{算例二:光滑初值}取初值函数\begin{equation}
u_0(x,y)=\frac{1+\sin{2\pi x}\sin {2\pi y}}{3}
\end{equation}
和参数$\theta = 4.0$.
\section{实验现象}
\begin{itemize}
    \item 对于算例二,newton法的CG收敛很慢,原因是正定矩阵条件数相比admm的CG太大,而admm的CG有$\rho$进行调节。
    \item 相同参数下,串行newton可能出现CG迭代不收敛的情况,并行版本不会,可能是因为并行版本不同的浮点数运算方法导致避免了某些数值误差
\end{itemize}
\section{算法优势}
猜测:在引入并行计算的模式下,由于ADMM的$U_2$子问题天然并行,受网格尺度影响更小,在更小的网格尺度下会体现出加速优势,不同时间步长下可以通过调整罚函数因子改进运算速度
\par 结果记录:取$dt=1e10$,$Nx=Ny=1024$,计算一个时间步,并行化程序用16个计算核心:注意当阈值取$1e-8$时能量下降才一致
\begin{itemize}
\item 随机初值
\begin{table}
    \centering
    \begin{tabular}{c|c|c|c|c|c}
    算法&判断收敛阈值&cpu\_time(s)&主迭代次数(次)& 能量 &残量\\
    \hline
    truncated\_admm & 1e-6 & 17.521 & admm:71 & 1.23249 & 2.04158e-05\\
    \hline 
    non\_cg\_admm  & 1e-6 & 18.912 & admm:26 & 1.23249 & 3.66484e-05\\
    \hline 
    non\_fft\_admm  & 1e-6 & 8.907 & admm:27 & 1.23249 & 2.91842e-05\\
    \hline
    newton & 1e-8 & 2.963 & newton:5 & 1.23249 & 1.85867e-05\\

    \end{tabular}
    \caption{随机初值数值结果}
    \label{随机初值数值结果}
\end{table}
\item 光滑初值
\begin{table}
    \centering
    \begin{tabular}{c|c|c|c|c|c}
    算法&判断收敛阈值&cpu\_time(s)&主迭代次数(次)& 能量 & 残量\\
    \hline
    truncated\_admm & 1e-6 & 15.496 & admm:82 & 0.595444 & 4.07057e-05\\
    \hline 
    non\_cg\_admm  & 1e-6 & 19.68 & admm:41 & 0.595444 & 3.30783e-05\\
    \hline 
    non\_fft\_admm  & 1e-6 & 10.935 & admm:42 & 0.595444 & 2.18749e-05\\
    \hline
    newton &  & 21.845 & newton:18 & 0.595444 & 1.02206e-05\\

    \end{tabular}
    \caption{光滑初值数值结果}
    \label{光滑初值数值结果}
\end{table}

\end{itemize}
\end{document}